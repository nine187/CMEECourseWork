\documentclass{article}
\usepackage[a4paper]{geometry}
\usepackage{graphicx}
\usepackage{float}

\linespread{1.5}
\title{Key West Temperature Autocorrelation}
\author{Dashing Dingos}
\date{Dec 2022}
\begin{document}

\section{Introduction}
Global warming is caused by anthropogenic impacts such as burning fossil fuels and deforestation, leading to increased carbon emission and climate level \cite{houghton2005global}.
Due to its geographic location, Florida have a high chance of facing the effects of climate change more severely than some other locations.
By using a dataset of previous climate records in Florida, we examine whether there has been a significant change in the temperatures of Florida over the years.
We examine one year's temperature and see if there is a significant correlation with the successive year's temperature.

\section{Methods}
We use the autocorrelation test with 10000 permutations to determine whether the correlation of the annual temperature difference is significant.
Firstly, we calculated the significance of the year's correlation with their successive year using the spearman method.
Then, we shuffled the dataset 10000 times and calculated the correlation coefficient each time.
We used a p-test to see the fraction of tests that are higher than the original value that was not being shuffled to determine whether there is a correlation.

\begin{figure}[H]
    \begin{center}
\centering
\includegraphics[scale= 0.5]{../results/TAuto2.png}
\caption{Correlation coefficient of 10000 permutation}
\end{center}
\end{figure}

\begin{figure}[H]
    \begin{center}
\centering
\includegraphics[scale= 0.5]{../results/TAuto1.png}
\caption{Temperature and year temperature dataset plot}
\end{center}
\end{figure}

\section{Results}
The observed correlation of the successive year is 0.341, and the p-value of the permutation test is 0.0004. 
The observed correlation of the successive year is significantly different from the distribution of the 10000 coefficient values being tested (Figure 1).


\section{Discussion}
The low p-value gathered could indicate that the pattern of climate changes observed occurs non-randomly.
The climate pattern could suggest a rise in temperature in Florida during the twentieth century.
The correlation result being non-random should be further looked into to determine whether the pattern is caused by climate change or other factors.

\bibliographystyle{apalike}
    \bibliography{TAutoCorr}
\end{document}